\subsection{Pitch push} % \label{app:...}
This test uses the test setup seen on \figref{fig:pitch_force}.

\subsection*{Test equipment:}
\begin{itemize}
\item Endowrist model 420093 (AAU number: \#4).
\item Maxon 110160 motor with attached Maxon gearhead 110356 and Maxon encoder 201937.
\item Load cell rate for 1 kg of force \cite{Load_cell_1kg}.
\item HX711 - Load cell amplifier \cite{HX711}
\end{itemize}

\subsection*{Procedure:}
The following procedure was made for the push force measurements:
\begin{enumerate}
\item The end-effector is rotated $90^\circ$ and attached perpendicular to the load cell. 
\item The scale is reset to zero.
\item Current is applied to the motor which control the pitch of the end-effector, with different current steps and the force is measured (downwards direction).
\item Current is increased until 1200 mA is applied.
\end{enumerate}
Step two to four is repeated five times, where the current and force is measured in respect to each other. 

\subsection*{Measuring data:}
The data from the measurements can be seen on \todo{Picture}%\figref{endo_force_mes}. 
%\eqref{eq:linear_force_endo}.

% \begin{equation}
% \text{y} = 0.0028 \cdot \text{x} -0.8259 
% \label{eq:linear_force_endo}
% \end{equation} 



\subsection*{Results:}
\todo{result}
%It can be seen from the graph on \figref{endo_force_mes} that the force on the end-effector is highly nonlinear. The friction from the gearing and the Endowrist does that the force first has an exponential growth at the start. Around the 800 mA and 1200 mA step it can be seen that a drop in force is happening. What causes this drop is not identified but it can be seen that it appears for all the data sequences. \todor{better explanation?}


%% This file was created by matlab2tikz.
%
%The latest updates can be retrieved from
%  http://www.mathworks.com/matlabcentral/fileexchange/22022-matlab2tikz-matlab2tikz
%where you can also make suggestions and rate matlab2tikz.
%
\definecolor{mycolor1}{rgb}{0.00000,0.44700,0.74100}%
\definecolor{mycolor2}{rgb}{0.85000,0.32500,0.09800}%
\definecolor{mycolor3}{rgb}{0.92900,0.69400,0.12500}%
\definecolor{mycolor4}{rgb}{0.49400,0.18400,0.55600}%
\definecolor{mycolor5}{rgb}{0.46600,0.67400,0.18800}%
\definecolor{mycolor6}{rgb}{0.30100,0.74500,0.93300}%
\definecolor{mycolor7}{rgb}{0.63500,0.07800,0.18400}%
%
\begin{figure}[H]
\begin{tikzpicture}
\centering
\begin{axis}[%
width=4.521in,
height=3.566in,
at={(0.758in,0.481in)},
scale only axis,
xmin=0,
xmax=1.25,
xlabel style={font=\color{white!15!black}},
xlabel={Ampere [A]},
ymin=0,
ymax=1.2,
ylabel style={font=\color{white!15!black}},
ylabel={Force [N]},
axis background/.style={fill=white},
title style={font=\bfseries},
title={Relationship between ampere and clamp force}
]
\addplot [color=mycolor1, draw=none, mark=asterisk, mark options={solid, mycolor1}, forget plot]
  table[row sep=crcr]{%
0.12	0.05892\\
0.24	0.12766\\
0.36	0.3437\\
0.48	0.6383\\
0.6	0.72668\\
0.72	0.82488\\
0.84	0.80524\\
0.96	0.92308\\
1.08	0.982\\
1.2	0.982\\
};
\addplot [color=mycolor2, draw=none, mark=asterisk, mark options={solid, mycolor2}, forget plot]
  table[row sep=crcr]{%
0.12	0.0491\\
0.24	0.16694\\
0.36	0.26514\\
0.48	0.53028\\
0.6	0.61866\\
0.72	0.67758\\
0.84	0.70704\\
0.96	0.8347\\
1.08	0.91326\\
1.2	0.90344\\
};
\addplot [color=mycolor3, draw=none, mark=asterisk, mark options={solid, mycolor3}, forget plot]
  table[row sep=crcr]{%
0.12	0.0491\\
0.24	0.12766\\
0.36	0.3437\\
0.48	0.62848\\
0.6	0.72668\\
0.72	0.79542\\
0.84	0.79542\\
0.96	0.89362\\
1.08	1.02128\\
1.2	1.00164\\
};
\addplot [color=mycolor4, draw=none, mark=asterisk, mark options={solid, mycolor4}, forget plot]
  table[row sep=crcr]{%
0.12	0.03928\\
0.24	0.0982\\
0.36	0.31424\\
0.48	0.64812\\
0.6	0.7365\\
0.72	0.80524\\
0.84	0.82488\\
0.96	0.92308\\
1.08	0.97218\\
1.2	0.99182\\
};
\addplot [color=mycolor5, draw=none, mark=asterisk, mark options={solid, mycolor5}, forget plot]
  table[row sep=crcr]{%
0.12	0.0491\\
0.24	0.12766\\
0.36	0.3437\\
0.48	0.67758\\
0.6	0.7856\\
0.72	0.84452\\
0.84	0.87398\\
0.96	0.982\\
1.08	1.00164\\
1.2	1.02128\\
};
\addplot [color=mycolor6, draw=none, mark=asterisk, mark options={solid, mycolor6}, forget plot]
  table[row sep=crcr]{%
0.12	0.0491\\
0.24	0.12766\\
0.36	0.35352\\
0.48	0.6874\\
0.6	0.79542\\
0.72	0.85434\\
0.84	0.90344\\
0.96	0.982\\
1.08	1.0311\\
1.2	1.0311\\
};
\addplot [color=mycolor7, draw=none, mark=asterisk, mark options={solid, mycolor7}, forget plot]
  table[row sep=crcr]{%
0.12	0.0491\\
0.24	0.11784\\
0.36	0.33388\\
0.48	0.66776\\
0.6	0.75614\\
0.72	0.81506\\
0.84	0.87398\\
0.96	0.97218\\
1.08	1.04092\\
1.2	1.0311\\
};
\end{axis}

\end{tikzpicture}%
\caption{Force measurements, measured at the end-effector with ampere increases of 120 mA }
\label{endo_force_mes}
\end{figure}

\subsection*{Uncertainties of measurement:}
\begin{itemize}
\item Not 100 \% orthogonal force to the load cell.
\item Input/Output impedance of sensors have a $\pm 10 \%$ tolerance.
\item Movement of test setup when force is generated.
\end{itemize}

\subsection*{Conclusion:}
\todo{conclusion}