\subsection{Pitch down and upwards force} % \label{app:...}
This test uses the test setup seen on \figref{fig:pitch_force}. The positive force is defined as a downwards direction on the load-cell, see \figref{fig:mes_up_down}.

\begin{figure}[H]
	\centering
	\includegraphics[width=0.4\linewidth]{load_cell_mes.png}
	\caption{Load-cell with defined upwards- downwars force.}
	\label{fig:mes_up_down}
\end{figure}


\subsection*{Test equipment:}
\begin{itemize}
\item Endowrist model 420093 (AAU number: \#4).
\item Maxon 110160 motor with attached Maxon gearhead 110356 and Maxon encoder 201937.
\item Load cell rate for 1 kg of force \cite{Load_cell_1kg}.
\item HX711 - Load cell amplifier \cite{HX711}.
\item Arduino uno with Max351 DAC.
\item sbRIO board.
\end{itemize}

\subsection*{Procedure:}
The following procedure was made:\\
Downwards force measurements:
\begin{enumerate}
\item The end-effector is attached perpendicular to the load cell. 
\item The load cell is reset to zero.
\item Current is applied to the motor which control the pitch of the end-effector, at different current levels and the force is measured (downwards direction).
\item Current is increased until 680 mA is applied.
\end{enumerate}
Step two to four is repeated five times, where the current and force is measured in respect to each other. 

Upwards force measurements:
\begin{enumerate}
\item The end-effector is attached perpendicular to the load cell. 
\item The load cell is reset to zero.
\item Current is applied to the motor which control the pitch of the end-effector, at different current levels and the force is measured (upwards direction).
\item Current is increased until 680 mA is applied.
\end{enumerate}
Step two to four is repeated five times, where the current and force is measured in respect to each other. 

\subsection*{Measuring data:}
Six of the data measurements can be seen on \figref{fig:pitch_down} and \figref{fig:pitch_up}.

\begin{figure}[H]
\centering
\input{Data/Measurement/EndoWrist_Measurements/Force/pitch_down}
\caption{Force measurements for the pitch in an downwards direction.}
\label{fig:pitch_down}
\end{figure}

\begin{figure}[H]
\centering
\input{Data/Measurement/EndoWrist_Measurements/Force/pitch_up}
\caption{Force measurements for the pitch in an upwards direction.}
\label{fig:pitch_up}
\end{figure}

%\figref{endo_force_mes}. 
%\eqref{eq:linear_force_endo}.

% \begin{equation}
% \text{y} = 0.0028 \cdot \text{x} -0.8259 
% \label{eq:linear_force_endo}
% \end{equation} 



\subsection*{Results:}
From \figref{fig:pitch_down} and \figref{fig:pitch_up} a similar pattern between the measurements can be seen. From each measurement it can be seen that the force growth for the different measurements are similar. Force is generated on the end-effector from almost the same current start point on both downwards and upwards measurements. For downwards the force is applied between -131 mA to -148 mA and upwards between 151 mA to 191 mA. It can be seen that the current is decreasing for each measurements. This is due to the controller implemented on the motor controller as it gets a setpoint and current is applied until the setpoint is reached and then decreased.
%It can be seen from the graph on \figref{endo_force_mes} that the force on the end-effector is highly nonlinear. The friction from the gearing and the Endowrist does that the force first has an exponential growth at the start. Around the 800 mA and 1200 mA step it can be seen that a drop in force is happening. What causes this drop is not identified but it can be seen that it appears for all the data sequences. \todor{better explanation?}


%% This file was created by matlab2tikz.
%
%The latest updates can be retrieved from
%  http://www.mathworks.com/matlabcentral/fileexchange/22022-matlab2tikz-matlab2tikz
%where you can also make suggestions and rate matlab2tikz.
%
\definecolor{mycolor1}{rgb}{0.00000,0.44700,0.74100}%
\definecolor{mycolor2}{rgb}{0.85000,0.32500,0.09800}%
\definecolor{mycolor3}{rgb}{0.92900,0.69400,0.12500}%
\definecolor{mycolor4}{rgb}{0.49400,0.18400,0.55600}%
\definecolor{mycolor5}{rgb}{0.46600,0.67400,0.18800}%
\definecolor{mycolor6}{rgb}{0.30100,0.74500,0.93300}%
\definecolor{mycolor7}{rgb}{0.63500,0.07800,0.18400}%
%
\begin{figure}[H]
\begin{tikzpicture}
\centering
\begin{axis}[%
width=4.521in,
height=3.566in,
at={(0.758in,0.481in)},
scale only axis,
xmin=0,
xmax=1.25,
xlabel style={font=\color{white!15!black}},
xlabel={Ampere [A]},
ymin=0,
ymax=1.2,
ylabel style={font=\color{white!15!black}},
ylabel={Force [N]},
axis background/.style={fill=white},
title style={font=\bfseries},
title={Relationship between ampere and clamp force}
]
\addplot [color=mycolor1, draw=none, mark=asterisk, mark options={solid, mycolor1}, forget plot]
  table[row sep=crcr]{%
0.12	0.05892\\
0.24	0.12766\\
0.36	0.3437\\
0.48	0.6383\\
0.6	0.72668\\
0.72	0.82488\\
0.84	0.80524\\
0.96	0.92308\\
1.08	0.982\\
1.2	0.982\\
};
\addplot [color=mycolor2, draw=none, mark=asterisk, mark options={solid, mycolor2}, forget plot]
  table[row sep=crcr]{%
0.12	0.0491\\
0.24	0.16694\\
0.36	0.26514\\
0.48	0.53028\\
0.6	0.61866\\
0.72	0.67758\\
0.84	0.70704\\
0.96	0.8347\\
1.08	0.91326\\
1.2	0.90344\\
};
\addplot [color=mycolor3, draw=none, mark=asterisk, mark options={solid, mycolor3}, forget plot]
  table[row sep=crcr]{%
0.12	0.0491\\
0.24	0.12766\\
0.36	0.3437\\
0.48	0.62848\\
0.6	0.72668\\
0.72	0.79542\\
0.84	0.79542\\
0.96	0.89362\\
1.08	1.02128\\
1.2	1.00164\\
};
\addplot [color=mycolor4, draw=none, mark=asterisk, mark options={solid, mycolor4}, forget plot]
  table[row sep=crcr]{%
0.12	0.03928\\
0.24	0.0982\\
0.36	0.31424\\
0.48	0.64812\\
0.6	0.7365\\
0.72	0.80524\\
0.84	0.82488\\
0.96	0.92308\\
1.08	0.97218\\
1.2	0.99182\\
};
\addplot [color=mycolor5, draw=none, mark=asterisk, mark options={solid, mycolor5}, forget plot]
  table[row sep=crcr]{%
0.12	0.0491\\
0.24	0.12766\\
0.36	0.3437\\
0.48	0.67758\\
0.6	0.7856\\
0.72	0.84452\\
0.84	0.87398\\
0.96	0.982\\
1.08	1.00164\\
1.2	1.02128\\
};
\addplot [color=mycolor6, draw=none, mark=asterisk, mark options={solid, mycolor6}, forget plot]
  table[row sep=crcr]{%
0.12	0.0491\\
0.24	0.12766\\
0.36	0.35352\\
0.48	0.6874\\
0.6	0.79542\\
0.72	0.85434\\
0.84	0.90344\\
0.96	0.982\\
1.08	1.0311\\
1.2	1.0311\\
};
\addplot [color=mycolor7, draw=none, mark=asterisk, mark options={solid, mycolor7}, forget plot]
  table[row sep=crcr]{%
0.12	0.0491\\
0.24	0.11784\\
0.36	0.33388\\
0.48	0.66776\\
0.6	0.75614\\
0.72	0.81506\\
0.84	0.87398\\
0.96	0.97218\\
1.08	1.04092\\
1.2	1.0311\\
};
\end{axis}

\end{tikzpicture}%
\caption{Force measurements, measured at the end-effector with ampere increases of 120 mA }
\label{endo_force_mes}
\end{figure}
\subsection*{Uncertainties of measurement:}
\begin{itemize}
\item The force applied is not perfectly orthogonal force to the load cell.
\item Input/Output impedance of load-cell sensors have a $\pm 10 \%$ tolerance.
\item Movement of test setup when force is generated.
\end{itemize}

\subsection*{Conclusion:}
Force is generated at the end-effector when a current higher than -131 mA for downwards force and 151 mA for upwards.